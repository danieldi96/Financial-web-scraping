\documentclass[12pt]{article}
 
\usepackage[utf8]{inputenc}
\usepackage{hyperref}
\usepackage[spanish]{babel}

\setlength {\marginparwidth }{2cm}
\usepackage[colorinlistoftodos]{todonotes}

\title{%
Práctica 1\\
\large Tipología y Ciclo de Vida de los Datos}
\author{Daniel Díaz Fernández}
\date{8 de noviembre de 2020}

\begin{document}
\maketitle

\section*{Contexto de los datos}
En el ámbito financiero, una 'bolsa' es un lugar en el que cotizan los productos financieros ligados, es decir, en el que se comercializa con acciones, ETF, futuros, etc. Cada día se publica la información de apertura y cierre de cualquier producto financiero. En el caso de España, la bolsa de referencia es la bolsa española de Madrid, la cual abre todos los días a las 9 y cierra a las 17:30 de la tarde. En ese transcurso horario, los productos financieros tienen una volatidad de mercado que se refleja en franjas de 5 minutos, dicho de otra manera, cada 5 minutos se publica su valro monetario.

La información que se ha recolectado proviene de las empresas que conforman el índice IBEX35, que es el índice de referencia en nuestro país. Ese índice corresponde a las 35 empresas con mayor capitalización bursátil. Además, esta información es accesible a cualquier ciudadano sea o no de España.

La página 'www.bolsamadrid.es' ha sido la escogida para la obtención de los datos, que es el sitio web oficial de la bolsa de Madrid.

Hay muchas otras fuentes (no oficiales) para acceder a los datos, aunque muchas de ellas son de pago o tienen un número limitado de peticiones por IP.

\section*{}
\paragraph{Tipología del dataset.\\}


\section*{}
\paragraph{Descripción del dataset.\\}


\section*{}
\paragraph{Representación gráfica.\\}


\section*{}
\paragraph{Contenido.\\}

\section*{}
\paragraph{Agradecimientos.\\}

\section*{}
\paragraph{Inspiración.\\}

\section*{}
\paragraph{Código.\\}

\section*{}
\paragraph{Dataset.\\}


\clearpage
\begin{thebibliography}{9}
	\bibitem{UOC_1}
	Subirats, L. Calvo, "Web Scraping", UOC (2018).
	
	\bibitem{UOC_2}
	Masip, D., "El lenguaje Python", UOC.

	
	
\end{thebibliography}

\end{document}
